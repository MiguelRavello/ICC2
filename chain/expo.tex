\documentclass[xcolor=dvipsnames]{beamer}

\usetheme{Antibes}
\usecolortheme[named=Maroon]{structure}
\usepackage[utf8]{inputenc}
\usepackage{default}
\usepackage{graphicx}
\usepackage{listings}
\usepackage{color}

\definecolor{gray97}{gray}{.97}
\definecolor{gray75}{gray}{.75}
\definecolor{gray45}{gray}{.45}
\lstset{ frame=Ltb,
     framerule=0pt,
     aboveskip=0.5cm,
     framextopmargin=3pt,
     framexbottommargin=3pt,
     framexleftmargin=0.4cm,
     framesep=0pt,
     rulesep=.4pt,
     backgroundcolor=\color{gray97},
     rulesepcolor=\color{black},
     %
     stringstyle=\ttfamily,
     showstringspaces = false,
     basicstyle=\small\ttfamily,
     commentstyle=\color{gray45},
     keywordstyle=\bfseries,
     %
     numbers=left,
     numbersep=15pt,
     numberstyle=\tiny,
     numberfirstline = false,
     breaklines=true,
   }

% minimizar fragmentado de listados
\lstnewenvironment{listing}[1][]
   {\lstset{#1}\pagebreak[0]}{\pagebreak[0]}

\lstdefinestyle{consola}
   {basicstyle=\scriptsize\bf\ttfamily,
    backgroundcolor=\color{gray75},
    }
\lstdefinestyle{C}
   {language=C,
   }

\title{Chain of Resposibility}
\author{por Ravelo Jove Juan Miguel}
\date{\today}

\begin{document}

\begin{frame}
 \titlepage
\end{frame}

\begin{frame}
\frametitle{Índice}
\tableofcontents[pausesections]

\end{frame}

\section{Definici\'on}
\begin{frame}
 \frametitle{Definici\'on}
 Una forma de pasar una solicitud entre una cadena de objetos.
  \begin{itemize}
 \item Solicitud.
 \item Manejadores.
 \item Lista enlazada.
 \item Cliente.
 \item Cajero autom\'atico.
  \end{itemize}
\end{frame}

\section{Intensi\'on}
\begin{frame}
 \frametitle{Intensi\'on}
\begin{itemize}
 \item Evita el emisor de una solicitud a su receptor, dando a mas de un objeto una oportunidad de manejar la solicitud. Encadena los objetos receptores y pasa la solicitud a lo largo de la cadena hasta que un objeto lo maneje.
 \item Solicitudes de \textbf{Launch and leave} con un \textbf{unico canal} de procesamiento que contiene muchos \textbf{manejadores}.
 \item Una \textbf{Lista enlazada} orientada a objetos con recorrido recursivo.
\end{itemize}
\end{frame}

\section{Problema}
\begin{frame}
  \frametitle{Problema}
\begin{itemize}
 \item Hay un n\'umero posible de \textbf{manejadores} o \emph{elementos de procesamiento} o \emph{nodos} y una secuencia de solicitudes a ser manejadas.
 \item Necesidad de procesar solicitudes de manera eficiente sin tener en cuenta las relaciones y prioridades del manejador, ni las solicitudes asignadas al manejador.
\end{itemize}
\end{frame}

\section{Encapsulamiento}
\begin{frame}
 \frametitle{Encapsulamiento}
 \begin{figure}
 \centering
 \includegraphics[scale=0.7]{./imag1.png}
 % imag1.png: 0x0 pixel, 300dpi, 0.00x0.00 cm, bb=
 \label{fig:1}
\end{figure}
\end{frame}

\section{Estructura}
\begin{frame}
 \frametitle{Estructura}
 \begin{figure}
 \centering
 \includegraphics[scale=0.7]{./imag2.png}
 % imag2.png: 0x0 pixel, 300dpi, 0.00x0.00 cm, bb=
 \label{fig:2}
\end{figure}

\end{frame}


\section{Cajero}
\begin{frame}[fragile]
 \frametitle{Header}
 \begin{lstlisting}[language=C++]
#ifndef CAJERO_H
#define CAJERO_H

#include<iostream>
using namespace std;

class Cajero{
protected:
    Cajero *m_next;
public:
    Cajero():m_next(NULL){}
    virtual void devuelveBilletes(int valor)=0;
    void setNext(Cajero *siguiente){    m_next=siguiente;}
};
  \end{lstlisting}
\end{frame}

\begin{frame}[fragile]
 \frametitle{Header}
 \begin{lstlisting}[language=C++]
class Billete50 : public Cajero{
public:
void devuelveBilletes(int valor){
  if(valor%5==0 || valor%10==0){
    int resto = valor%50;
    int nbilletes = valor/50;
    if(nbilletes>0)
      cout<<nbilletes<<" x50 soles"<<endl;      
  \end{lstlisting}
\end{frame}

\begin{frame}[fragile]
 \frametitle{Header}
 \begin{lstlisting}[language=C++]
    if(resto>0){
      if(m_next!=NULL)
	m_next->devuelveBilletes(resto);
      else
	cout<<"soy el ultimo handler"<<endl;
    }
  }
  else
    throw "valor no valido";
}
};
 \end{lstlisting}
\end{frame}


\begin{frame}[fragile]
 \frametitle{Header}
 \begin{lstlisting}[language=C++]
class Billete20 : public Cajero{
public:
    void devuelveBilletes(int valor){
        int resto = valor%20;
        int nbilletes = valor/20;
        if(nbilletes>0)
            cout<<nbilletes<<" x20 soles"<<endl;
        if(resto>0){
            if(m_next!=NULL)
                m_next->devuelveBilletes(resto);
            else
                cout<<"soy el ultimo handler"<<endl;
        }
    }
};
  \end{lstlisting}
\end{frame}

\begin{frame}[fragile]
 \frametitle{Main}
 \begin{lstlisting}[language=C++]
#include"cajero.h"
int main(){
    Cajero *b50 = new Billete50();
    Cajero *b20 = new Billete20();
    Cajero *b10 = new Billete10();
    Cajero *b5  = new Billete5();
//  Enlazando los handlers
    b50->setNext(b20);
    b20->setNext(b10);
    b10->setNext(b5);
  \end{lstlisting}
\end{frame}

\begin{frame}[fragile]
 \frametitle{Main}
 \begin{lstlisting}[language=C++]
//  Red de seguridad contra numeros no multiplos de 5
    try{
        b50->devuelveBilletes(23); // set la solicitud
    } catch(const char* msg){
        cerr << msg << endl;
    }
    delete b50;
    delete b20;
    delete b10;
    delete b5;
    return 0;
}
  \end{lstlisting}
\end{frame}

\section{Conclusiones}
\begin{frame}
 \frametitle{Conclusiones}
  Reduce el acoplamiento.
 \begin{itemize}
  \item El patr\'on libera a un objeto de tener que saber que otro objeto maneja la solicitud.
  \item El receptor (manejador) no conoce al emisor.
  \item El objeto (manejador) no conoce la estructura de la cadena (pipeline).
  \item Simplifica las interconexiones entre objetos, en vez que los objetos mantengan referencia a posibles receptores solo refiere al siguiente.
 \end{itemize}
\end{frame}

\begin{frame}
 \frametitle{Conclusiones}
 A\~nade flexibilidad para asignar resposabilidad a objetos. \\
 
 No se garantiza la recepci\'on, dado que las solicitudes no tienen un receptor explicito.
\end{frame}


\end{document}