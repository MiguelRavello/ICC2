\documentclass[a4paper,10pt]{article}
\usepackage[utf8]{inputenc}
\usepackage{listings}
\usepackage{graphicx}
\usepackage{color}

\definecolor{gray97}{gray}{.97}
\definecolor{gray75}{gray}{.75}
\definecolor{gray45}{gray}{.45}
\lstset{ frame=Ltb,
     framerule=0pt,
     aboveskip=0.5cm,
     framextopmargin=3pt,
     framexbottommargin=3pt,
     framexleftmargin=0.4cm,
     framesep=0pt,
     rulesep=.4pt,
     backgroundcolor=\color{gray97},
     rulesepcolor=\color{black},
     %
     stringstyle=\ttfamily,
     showstringspaces = false,
     basicstyle=\small\ttfamily,
     commentstyle=\color{gray45},
     keywordstyle=\bfseries,
     %
     numbers=left,
     numbersep=15pt,
     numberstyle=\tiny,
     numberfirstline = false,
     breaklines=true,
   }

% minimizar fragmentado de listados
\lstnewenvironment{listing}[1][]
   {\lstset{#1}\pagebreak[0]}{\pagebreak[0]}

\lstdefinestyle{consola}
   {basicstyle=\scriptsize\bf\ttfamily,
    backgroundcolor=\color{gray75},
    }
\lstdefinestyle{C}
   {language=C,
   }
%\hyperlinking
%opening
\title{Chain of Resposibility}
\author{por Juan Miguel Ravelo Jove}

\begin{document}

\maketitle

\paragraph{Palabras clave:} 
\begin{itemize}
 \item Solicitud.
 \item Manejadores.
 \item Lista enlazada.
 \item Cliente.
 \item Cajero autom\'atico.
\end{itemize}

\section{Definici\'on del patr\'on}
Una forma de pasar una solicitud entre una cadena de objetos.

\section{Explicaci\'on}
\subsection{Intensi\'on}
\begin{itemize}
 \item Evita el emisor de una solicitud a su receptor, dando a mas de un objeto una oportunidad de manejar la solicitud. Encadena los objetos receptores y pasa la solicitud a lo largo de la cadena hasta que un objeto lo maneje.
 \item Solicitudes de \textbf{Launch and leave} con un \textbf{unico canal} de procesamiento que contiene muchos \textbf{manejadores}.
 \item Una \textbf{Lista enlazada} orientada a objetos con recorrido recursivo.
\end{itemize}

\subsection{Problema}
\begin{itemize}
 \item Hay un n\'umero posible de \textbf{manejadores} o \emph{elementos de procesamiento} o \emph{nodos} y una secuencia de solicitudes a ser manejadas.
 \item Necesidad de procesar solicitudes de manera eficiente sin tener en cuenta las relaciones y prioridades del manejador, ni las solicitudes asignadas al manejador.
\end{itemize}

\subsection{An\'alisis y discuciones}
Encapsular los elementos de procesamiento dentro de una abstracci\'on de \textbf{\emph{canalizaci\'on}} y hacer que los clientes \textbf{launch and leave} sus solicitudes a la entrada de la \textbf{pipeline}. \\
%imagen del patron de diseno
\begin{figure}[!h]
 \centering
 \includegraphics[scale=0.7]{./imag1.png}
 % imag1.png: 0x0 pixel, 300dpi, 0.00x0.00 cm, bb=
 \caption{Encapsulamiento en una pipeline}
 \label{fig:1}
\end{figure}

El patr\'on encadena los objetos entre si, y luego pasa los mensajes de solicitud de un objeto a otro hasta que alcanza un objeto capaz de manejar el mensaje. La cantidad y el tipo de objetos del manejador no se conocen a priori, se pueden configurar din\'amicamente. El mecanismo de encadenamiento utiliza una composici\'on recursiva para permitir un \textbf{n\'umero ilimitado} de manejadores a \textbf{vincular}. \\

Chain of resposibility simplifica las interconexiones de objetos. En lugar de remitentes y receptores que mantienen referencias a todos los receptores candidatos, \textbf{\emph{cada emisor guarda una sola referencia a su sucesor inmediato en la cadena}}. \\

Aseg\'urese de que exista una \textbf{red de seguridad para atrapar} a cualquier solicitud que no se haya manejado.
\paragraph{Cuidado}
No use Chain of resposibility cuando cada solicitud solo sea manejada por un solo manejador, o cuando el objeto del cliente sepa que objeto servicio debe manejar la solicitud.

\subsection{Casos donde usar}
\'Usese el patr\'on Chain of responsibility cuando:
\begin{itemize}
 \item Hay m\'as de un objeto que pueda manejar la solicitud y el manejador no se conoce a priori, sino que deber\'ia determinarse autom\'aticamente.
 \item Se quiere enviar una solicitud a un objeto entre varios, sin especificar explicitamente el receptor.
 \item El conjunto de objetos que pueden manejar una solicitud deberia ser especificado din\'amicamente.
\end{itemize}

\subsection{Estructura}
\begin{itemize}
 \item Las clases derivadas saben como satisfacer las solicitudes de los clientes.
 \item Si el objeto \textbf{actual} no esta disponible, o no es suficiente, se delega en la clase base, que delega en el objeto \textbf{siguiente}, y continuan las operaciones.
 \item Multiples manejadores podrian contribuir al manejo de cada solicitud.
 \item La solicitud puede transmitirse a lo largo de toda la cadena, teniendo cuidado que el ultimo enlace, un \textbf{NULL siguiente}. 
\end{itemize}

\begin{figure}[h!]
 \centering
 \includegraphics[scale=0.7]{./imag2.png}
 % imag2.png: 0x0 pixel, 300dpi, 0.00x0.00 cm, bb=
 \caption{Estructura}
 \label{fig:2}
\end{figure}


\section{Ejemplo de un cajero autom\'atico}
\paragraph{Header}
\begin{lstlisting}[language=C++]
#ifndef CAJERO_H
#define CAJERO_H

#include<iostream>
using namespace std;

class Cajero{
protected:
    Cajero *m_next;
public:
    Cajero():m_next(NULL){}
    virtual void devuelveBilletes(int valor)=0;
    void setNext(Cajero *siguiente){    m_next=siguiente;}
};

class Billete50 : public Cajero{
public:
    void devuelveBilletes(int valor){
        if(valor%5==0 || valor%10==0){
            int resto = valor%50;
            int nbilletes = valor/50;
            if(nbilletes>0)
                cout<<nbilletes<<" x50 soles"<<endl;
            if(resto>0){
                if(m_next!=NULL)
                    m_next->devuelveBilletes(resto);
                else
                    cout<<"soy el ultimo handler"<<endl;
            }
        }
        else
            throw "valor no valido";
    }
};

class Billete20 : public Cajero{
public:
    void devuelveBilletes(int valor){
        int resto = valor%20;
        int nbilletes = valor/20;
        if(nbilletes>0)
            cout<<nbilletes<<" x20 soles"<<endl;
        if(resto>0){
            if(m_next!=NULL)
                m_next->devuelveBilletes(resto);
            else
                cout<<"soy el ultimo handler"<<endl;
        }
    }
};

class Billete10 : public Cajero{
public:
    void devuelveBilletes(int valor){
        int resto = valor%10;
        int nbilletes = valor/10;
        if(nbilletes>0)
            cout<<nbilletes<<" x10 soles"<<endl;
        if(resto>0){
            if(m_next!=NULL)
                m_next->devuelveBilletes(resto);
            else
                cout<<"soy el ultimo handler"<<endl;
        }
    }
};

class Billete5 : public Cajero{
public:
    void devuelveBilletes(int valor){
        int resto = valor%5;
        int nbilletes = valor/5;
        if(nbilletes>0)
            cout<<nbilletes<<" x5 soles"<<endl;
        if(resto>0){
            if(m_next!=NULL)
                m_next->devuelveBilletes(resto);
            else
                cout<<"soy el ultimo handler"<<endl;
        }
    }
};

#endif
\end{lstlisting}

\paragraph{Main}
\begin{lstlisting}[language=C++]
 #include"cajero.h"

int main(){
    Cajero *b50 = new Billete50();
    Cajero *b20 = new Billete20();
    Cajero *b10 = new Billete10();
    Cajero *b5  = new Billete5();
//  Cajero *b5 = NULL;  // el ultimo handler

//  Enlazando los handlers
    b50->setNext(b20);
    b20->setNext(b10);
    b10->setNext(b5);

//  Red de seguridad contra numeros no multiplos de 5
    try{
        b50->devuelveBilletes(23); // set la solicitud
    } catch(const char* msg){
        cerr << msg << endl;
    }

    delete b50;
    delete b20;
    delete b10;
    delete b5;

    return 0;
}
\end{lstlisting}

\section{Conclusiones}
\begin{enumerate}
 \item Reduce el acoplamiento.
 \begin{itemize}
  \item El patr\'on libera a un objeto de tener que saber que otro objeto maneja la solicitud.
  \item El receptor (manejador) no conoce al emisor.
  \item El objeto (manejador) no conoce la estructura de la cadena (pipeline).
  \item Simplifica las interconexiones entre objetos, en vez que los objetos mantengan referencia a posibles receptores solo refiere al siguiente.
 \end{itemize}
 \item A\~nade flexibilidad para asignar resposabilidad a objetos.
 \item No se garantiza la recepci\'on, dado que las solicitudes no tienen un receptor explicito.
\end{enumerate}

\section{Referencias}
\begin{itemize}
 \item \texttt{https://sourcemaking.com/design\_patterns/chain\_of\_responsibility/cpp/1}
 \item \texttt{http://www.coderslexicon.com/chain-of-responsibility-pattern-c/}
 \item Alexander Shvets, Design Patterns Explained Simply (2013).
 \item Erich Gamma, Patrones de Dise\~no 1ra edici\'on (2003).
\end{itemize}

\end{document}